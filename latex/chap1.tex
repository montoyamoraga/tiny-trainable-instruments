%% This is an example first chapter.  You should put chapter/appendix that you
%% write into a separate file, and add a line \include{yourfilename} to
%% main.tex, where `yourfilename.tex' is the name of the chapter/appendix file.
%% You can process specific files by typing their names in at the 
%% \files=
%% prompt when you run the file main.tex through LaTeX.
\chapter{Introduction}

Cras nec mauris feugiat, aliquam elit ac, blandit ex \cite{article-full}.

\section{Section sample}\label{ch1:sec}

Nulla sed sem finibus, vehicula quam at, vulputate tellus\footnote{Here is a sample footnote referencing figures~\ref{arm:fig1}
and~\ref{arm:fig2}.}  

\subsection{Subsection sample}

% This is an example of how you would use tgrind to include an example
% of source code; it is commented out in this template since the code
% example file does not exist. To use it, you need to remove the '%' on the
% beginning of the line, and insert your own information in the call.
%
%\tagrind[htbp]{code/pmn.s.tex}{Code sample}{opt:pmn}

Donec blandit dolor a ipsum sodales, eget aliquet nisl fermentum.

\begin{enumerate}
  \item Item 1.
\end{enumerate}

% This is an example of how you would use tgrind to include an example
% of source code; it is commented out in this template since the code
% example file does not exist.  To use it, you need to remove the '%' on the
% beginning of the line, and insert your own information in the call.
%
%\tgrind[htbp]{code/be.s.tex}{Block Exponent}{opt:be}

\subsection{Another subsection sample}

This is done by using some combination of

\begin{eqnarray*}
a_i & = & a_j + a_k \\
a_i & = & 4a_j + a_k \\
a_i & = & a_j \ll m \mbox{shift}
\end{eqnarray*}
