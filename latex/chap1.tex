%% This is an example first chapter.  You should put chapter/appendix that you
%% write into a separate file, and add a line \include{yourfilename} to
%% main.tex, where `yourfilename.tex' is the name of the chapter/appendix file.
%% You can process specific files by typing their names in at the 
%% \files=
%% prompt when you run the file main.tex through LaTeX.
\chapter{Introduction}

Lorem ipsum dolor sit amet, consectetur adipiscing elit. Aliquam quis sollicitudin metus. Quisque quam ex, tincidunt et porttitor quis, tincidunt faucibus quam. Nulla facilisi. Nam a libero posuere, mattis leo ac, ultrices est. Nullam auctor lacus eu metus venenatis, gravida consectetur felis laoreet. Nam non ante felis. Maecenas id dignissim turpis, eget pulvinar nisl. Cras nec mauris feugiat, aliquam elit ac, blandit ex \cite{article-full}.

\section{Section sample 1}

Ut hendrerit risus egestas, sollicitudin mauris sit amet, fermentum ipsum. Donec vulputate enim in justo pellentesque rhoncus. Nunc a dui condimentum, egestas ipsum eu, fermentum enim. Duis condimentum iaculis luctus. Nam sodales pellentesque luctus. Aenean tristique ante mattis tellus tincidunt, vitae mattis nunc tristique. Ut nec mattis ante, eu sodales ex.

\section{Section sample 2}\label{ch1:sec}

Sed vel lectus ut dui tempor molestie. Suspendisse blandit sapien posuere quam tempor lobortis. Duis sollicitudin tincidunt dui, at aliquam lorem dictum sit amet. Aenean congue nibh lectus, ut faucibus turpis facilisis quis. Ut aliquet magna at placerat ultricies. Mauris convallis, risus efficitur gravida dapibus, lacus lorem malesuada ligula, eget porta diam felis non turpis. Nulla sed sem finibus, vehicula quam at, vulputate tellus\footnote{Here is a sample footnote referencing figures~\ref{arm:fig1}
and~\ref{arm:fig2}.}  

\subsection{Subsection sample}

% This is an example of how you would use tgrind to include an example
% of source code; it is commented out in this template since the code
% example file does not exist. To use it, you need to remove the '%' on the
% beginning of the line, and insert your own information in the call.
%
%\tagrind[htbp]{code/pmn.s.tex}{Code sample}{opt:pmn}

Pellentesque ac leo eget lorem vulputate mattis eu a nisl. Duis elit erat, consectetur vulputate ullamcorper a, finibus quis turpis. Vivamus tincidunt dui id purus bibendum malesuada. Fusce accumsan, ipsum quis feugiat sodales, enim est aliquet leo, ut ornare justo mauris quis ex. Sed eros magna, suscipit et blandit non, pretium id felis. Praesent a vehicula tortor. Donec blandit dolor a ipsum sodales, eget aliquet nisl fermentum.

\begin{enumerate}
  \item Item 1.
  \item Item 2.
  \item Item 3.
\end{enumerate}

% This is an example of how you would use tgrind to include an example
% of source code; it is commented out in this template since the code
% example file does not exist.  To use it, you need to remove the '%' on the
% beginning of the line, and insert your own information in the call.
%
%\tgrind[htbp]{code/be.s.tex}{Block Exponent}{opt:be}

\subsection{Another subsection sample}

Sed quis dapibus libero. Curabitur id finibus nulla, sed semper felis. Proin dapibus nulla interdum, bibendum tortor et, blandit sapien. Etiam pretium tristique tortor non lacinia. Aliquam dapibus turpis lorem, sit amet porta ex dignissim vitae. In neque felis, sagittis sed ullamcorper lacinia, lobortis ut turpis. Nam quis aliquet justo. Nam eros mi, aliquam vel massa ac, ornare dignissim erat.  This is done by using some combination of
\begin{eqnarray*}
a_i & = & a_j + a_k \\
a_i & = & 2a_j + a_k \\
a_i & = & 4a_j + a_k \\
a_i & = & 8a_j + a_k \\
a_i & = & a_j - a_k \\
a_i & = & a_j \ll m \mbox{shift}
\end{eqnarray*}
