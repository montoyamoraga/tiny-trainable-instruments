\chapter{Conclusions}

TODO: the conclusions should be somewhat standalone in letting a reader start at the conclusion and understand the main topics but more importantly get a sense of the big takeaways without needing to go back to the internal chapters. they should also be compelling enough that they may pique someone's interest to go back to earlier chapters if they've started out at the end.
TODO: shorter hardware section, extend future connections, add big impact of this project, and why it matters

During this thesis I have presented the design and the complete process behind the development of interactive standalone instruments for manipulation of multimedia material using machine learning. The creation of these instruments involved designing hardware, publishing a software library, writing and teaching workshops, and making new art performances and workflows with these instruments.

TODO: add why this matters
computers could be evil, backdoors, the cloud, oh no
machine learning is scary
power to the people
add example of facial recognition banning~
UTOPIA: machine learning doesn’t exploit people, it gives more answers than questions


During this journey, I discussed different strategies for media arts education, ethics in artificial intelligence, and design of hardware and software tools for artists. This work is a foundation for further research, including the creation of subsequent multimedia instruments and software, and the creation of new courses and educational units for arts, computer science, and interaction design.
TODO: what is the big “so what?” what is the impact?

how i am connecting my work with future work
TODO: add the voices of people who go to the workshops~

\section{Lessons learned}

so that future makers can learn from this process

Work with other people

Dont’ assume things

maybe appendix: 3 or 5 things i could do differenntly
maybe appendix: what could be different in a future without covid

\section{Contributions}


\section{Future work}

\subsection{Hardware for new instruments}

TODO: right now this section is not a brief summary of what i’ve presented before, it is too detailed. Instead, quickly summarize what you've discussed in each chapter in no more than one paragraph per chapter and then switch to the impact of your work and future connections.

This project relies on a particular microcontroller, the Arduino Nano 33 BLE Sense. I used and Arduino board because of its popularity, availability, and software and community support.

The particular board's main draw for building multimedia instruments is its embedded sensors, including microphone, gyroscope, and camera. This embedding on the chip makes it more attractive than other boards, where users need to acquire, wire, and calibrate off-board sensors, making the process of capturing data more cumbersome and expensive in resources, adding additional barriers to instrument makers and prototypers. 

When this project started in 2020, and still today, this board was the only Arduino microcontroller supported by the Google TensorFlow Lite Micro library, for machine learning using microcontrollers, and it was heavily featured on the first promotional and educational material, which were the direct inspiration for this thesis, in terms of openness about the materials.

Microcontrollers come and go, most probably this board will be discontinued, and I hope this project can be adapted to other boards and architectures, particularly to other open source microcontrollers, including other Arduino boards, PJRC Teensy and Adafruit Circuit Playground, among others.

In terms of the outputs of the Tiny Trainable Instruments, I focused on creating many parallel multimedia approaches, including making sounds with piezo buzzers and MIDI, manipulating light with LEDs, creating movement and rhythm with servo motors, and printing text with thermal printers and screens. This is to appeal to a larger audience of artists and learners, interested in different mediums.

The Tiny Trainable Instruments are built with prototyping electronic breadboards, to make explicit their open-endedness, and to relate to a flux, instead of a fixed in state PCB, and a further state of this project would include the creation of custom boards with fixed wiring, and of enclosures and packaging.


\subsection{Software for new instruments}

This thesis has been published as an open source software library for Arduino. It promotes modularity and adaptability, where a Tiny Trainable Instrument can be any combination of the multiple inputs and outputs.

The file structure of the source code and the software dependencies of this library was also built with flexibility in mind, to encourage the remix and adaptation of this library to further projects.

A particular challenging aspect of this project, is the breadth of the disciplines combined, and its novel application of machine learning in microcontrollers. As discussed in previous chapters, there is a trend and new wave of builders and makers creating standalone multimedia instruments, based on open operating systems like Linux, and/or different microcontrollers. 

Despite the existence of artists and makers building standalone computational instruments, these skills are still hard to acquire. Additionally, the principles of this project, including being as cheap as possible, and as open as possible, are designed to encourage experimentation and hacking, but also can pose additional challenges. I hope this project encourages people to learn how to make instruments, and also engages in discourse about the creation of new curricula for the next generation of instrument makers and artists.

Another particular challenging aspect of writing software for multimedia instruments includes the licensing. The dependencies of this software are mostly other libraries by Arduino, Google, Adafruit, and with different licenses including public domain, MIT, and Apache. I hope this document helps to navigate these legal complexities and that this project helps artists and enthusiasts to navigate this landscape and overcome these barriers.

\subsection{Education, courses workshops}

I hope that this thesis project is adopted by educators, to introduce students to machine learning, physical computing, media arts, and ethics.

Many sections of this project could be adapted to further existing curricula for music, rhythm, ethics, computer science, and to create a new wave of instrument makers and media artists.

TODO: maybe add potential future users. Talk about how I envision this project being used in these contexts.
