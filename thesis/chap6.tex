\chapter{Conclusions and future work}

\epigraph{Don't get any big ideas \\ they're not gonna happen}{Nude \\ Radiohead, 2007}

In this thesis I have presented all the stages of the design and development of new standalone multimedia instruments using machine learning and microcontrollers, emphasizing \acrshort{AI} ethics. The project includes software examples, hardware suggestions, educational material, and strategies for ethical off-cloud machine learning and creation of custom artisanal databases.

This thesis is also the basis for further research, including the creation of subsequent multimedia instruments and software libraries, the writing of new courses and educational units at the intersection of arts, physical computing, interaction design, and computational ethics.

\section{Contributions}

Concrete contributions:

\begin{enumerate}
  \item Publishing TinyTrainable, a software library for \acrshort{ML} with microcontrollers for multimedia art.
  \item Design, writing, and teaching a 4-hour workshop for beginners, enthusiasts, and artists to teach with the software library.
  \item Publishing code and tutorials for creating custom databases for gesture and speech.
  \item Publishing custom-trained machine learning models for gesture and speech recognition.
  \item Publishing code and tutorials for training machine learning algorithms on the cloud and on personal computers for privacy and agency.
  \item Publishing other related software libraries, such as MaquinitasParams for communication with other instruments, and MaquinitasRitmos for rhythmic data.
\end{enumerate}

Abstract contributions:

\begin{enumerate}
 \item Demonstrating how a broader range of people can use machine learning to support their creative expression.
\end{enumerate}

\section{Lessons learned}

\begin{enumerate}
  \item Writing software for artists is hard.
  \item Writing software libraries for other artists is even harder.
  \item Collaboration with other people is essential to write code and educational material.
  \item Documentation of all design decisions is key to explaining why and how everything works.
  \item Navigating licenses and copyright is hard.
\end{enumerate}

\section{Future work}

\subsection{Hardware for new instruments}

This thesis relies on an Arduino microcontroller because of their open source nature, commercial availability, software and community support, and detailed documentation.

In particular I picked the Arduino Nano 33 \acrshort{BLE} Sense, because of two main reasons at the time this project started in late 2020: it is the only Arduino supported by TensorFlow Lite for microcontrollers and featured in the HarvardX certificate on Tiny Machine Learning, and also because of its convenience of having embedded sensors, which makes it simpler and cheaper to acquire data for live interaction and for building custom databases, eliminating barriers to instrument makers and prototypers.

Microcontrollers come and go. Most probably this board will be discontinued, but the strategies and software can be forked and adapted to other microcontrollers and software architectures. I am particularly looking in this thesis project to other open source microcontrollers, including other Arduino boards, PJRC Teensy and Adafruit Circuit Playground, which would enable the adoption of other software stacks, such as Python instead of C++, and also to other communities building multimedia instruments and experiences.

In terms of the outputs of the Tiny Trainable Instruments, I focused on creating many parallel multimedia approaches, including making sounds with piezo buzzers and MIDI, manipulating light with LEDs, creating movement and rhythm with servo motors, and printing text with thermal printers and screens. This is to appeal to a larger audience of artists and learners, interested in different mediums, and I hope this thesis project inspires more complex artworks and interactive experiences that this library currently allows, and that people can contribute back to the library to share these new capabilities with everyone.

The Tiny Trainable Instruments are built with prototyping electronic breadboards, to make explicit their open-endedness, and to promote experimentation and lower barriers. A further iteration of this project would include the creation of custom printed circuit boards with fixed wiring, and also enclosures and packaging.

\subsection{Software for new instruments}

This thesis has been published as an open source software library for Arduino. It promotes modularity and adaptability, where a Tiny Trainable Instrument can be any combination of the multiple inputs and outputs.

The file structure of the source code and the software dependencies of this library were also built with flexibility in mind, to encourage the remix and adaptation of this library to future projects.

A challenging aspect of this project is the breadth of the disciplines combined, and its novel application of machine learning in microcontrollers. As discussed in previous chapters, there is a trend and new wave of builders and makers creating standalone multimedia instruments, based on open operating systems like Linux, and/or different microcontrollers. 

Despite the existence of artists and makers building standalone computational instruments, these skills are still hard to acquire. Additionally, the principles of this project, including being as cheap as possible, and as open as possible, are designed to encourage experimentation and hacking, but also can pose additional challenges. I hope this project encourages people to learn how to make instruments, and also engages in discourse about the creation of new curricula for the next generation of instrument makers and artists.

Another challenging aspect of writing software for multimedia instruments is its licensing, both choosing a license and also respecting and understanding the license of other code and resources we are using. The dependencies of this software are mostly other libraries by Arduino, Google, Adafruit, and with different licenses including public domain, MIT, and Apache. I hope this document helps to navigate these legal complexities and that this project helps artists and enthusiasts to navigate this landscape and overcome these barriers.

\subsection{Educational impact}

This project was built to inspire and celebrate a new generation of coursework, workshops, and books, in the disciplines of ethical machine learning and microcontroller-based instruments.

I hope that this thesis project is adopted by educators, to introduce students to machine learning, physical computing, media arts, and computational ethics. It would be amazing if aspects of Tiny Trainable Instruments could be incorporated into existing music, arts, sculpture, and computer science curricula, to create a new wave of instrument makers and media artists.
