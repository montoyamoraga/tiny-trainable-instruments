\chapter{Scripts}

\epigraph{By pressing down a special key, it plays a little melody}{Pocket calculator \\ Kraftwerk, 1981}

During the writing of this thesis I developed scripts, which are included in this repository on the folder scripts.

I abstracted them to make them useful for other people and published them with a MIT License at https://github.com/montoyamoraga/scripts.

Since they are useful scripts and they are commented, I include them here.

\section{Formatting code with clang-format}

clang-format is a command line tool for formatting code. This script was written to auto format the code from the Arduino/C++ library TinyTrainable.

\lstinputlisting[language=bash]{../scripts/clang-format.sh}

\section{Converting formats with ffmpeg}

ffmpeg is a command line tool for converting audiovisual files between formats. This script was written to convert audio files from .mp3 to .ogg format for training a database for speech recognition.

\lstinputlisting[language=bash]{../scripts/ffmpeg-convert.sh}

\section{Deleting metadata with exiftool}

exiftool is a command line tool for reading and writing metadata from files. This script was written to delete metadata from images, like GPS coordinates added by modern smartphones, only keeping the actual image.

\lstinputlisting[language=bash]{../scripts/exiftool-delete-metadata.sh}

\section{Converting formats with pandoc}

pandoc is a command line tool for converting between formats. This script was written to convert from .tex files to .docx files, so that each chapter of this thesis document could be uploaded to Google Docs for feedback from the committee.

\lstinputlisting[language=bash]{../scripts/pandoc-convert.sh}
